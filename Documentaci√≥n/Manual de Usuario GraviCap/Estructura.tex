\chapter{Estructura}
    \section{Materiales}
        \begin{itemize} [label=•]
        \setlength{\itemindent}{3em}
            \item 3 varas de 87 cm de largo y 8 milímetros de grosor.
            \item 1 vara en forma L de 82 cm de largo y 3 milímetros de grosor.
            \item 2 caños de 230 cm de largo y 7,5 cm de ancho. 
            \item 2 estructuras de caños de 98 cm de largo y 89 cm de alto. La superior tiene una polea y la estructura inferior tiene dos poleas.
            \item Placa MPPT.
            \item Placa del microcontrolador del MPPT.
            \item Placa step down.
            \item Placa etapa de control.
            \item Fuente de energía renovable (para este prototipo es un panel solar de 24 voltios).
            \item 6 metros de cable de acero sin recubrimiento de 6 milímetros de grosor.
            \item 1 peso de 30 kilogramos de cemento.
            \item 1 motor.
            \item 1 encoder.
        \end{itemize}
        
    \section{Herramientas}
        \begin{itemize} [label=•]
        \setlength{\itemindent}{3em}
            \item Llave Alem de 6mm.
            \item Llave de 17mm.
            \item Llave de 8mm.
            \item Soportes para el peso.
        \end{itemize}

    \section{Precauciones} 
        Al manipular o mover el peso de la batería se recomienda moverlo entre 2 personas para llevarlo más fácilmente, sugerimos la utilización de calzado de seguridad, guantes y un equipo adecuado para movilizarlo, como una carretilla o carro. Se debe manejar con cuidado para evitar la caída del peso, este puede causar lesiones en personas circundantes o los pisos.\par
        Asegurarse de que la energía generada por la batería sea adecuada para el dispositivo que requiera cargar, para evitar una posible sobrecarga en el dispositivo conectado, quemando o dañando los componentes que lo conforman.\par
        Verificar los valores de voltaje y/o corriente de forma periódica para estar al tanto de cambios repentinos o paulatinos en el funcionamiento del dispositivo.\par
        No realizar mantenimiento en la estructura mientras el cable esté colocado con el peso acoplado.
        Verificar que el cable de acero esté tensado de manera correcta y bien acoplado en el sistema de poleas.\par
        Al armar y desarmar la estructura se debe preveer un soporte para el peso, apoyándolo suavemente por medio de una descarga mientras le quita tensión al cable.\par