\chapter{Limitaciones y Próximos Pasos}

    \section{Limitaciones}
        Durante el desarrollo del proyecto \textcolor{dark_violet}{GraviCap}, nos enfrentamos a varias limitaciones propias de trabajar con recursos restringidos y bajo las condiciones de un grupo de estudiantes en una escuela secundaria técnica. Estas limitaciones afectaron tanto el diseño como la implementación del prototipo, pero también nos ofrecieron valiosas lecciones sobre cómo escalar y mejorar el sistema en el futuro.\par
        
        \subsection{Capacidad de Almacenamiento de Energía}
        
            Una de las principales limitaciones del prototipo es la capacidad de almacenamiento de energía, que está determinada principalmente por la masa utilizada y la altura a la que puede ser elevada. El sistema actual utiliza una masa de 27 kg y una altura de 2,2 metros, lo que permite almacenar una cantidad limitada de energía potencial gravitatoria. Aunque esto es suficiente para demostrar el principio de funcionamiento del sistema, resulta insuficiente para aplicaciones a gran escala. La fórmula de la energía potencial, $E_p = mgh$, muestra que, para aumentar la capacidad de almacenamiento, sería necesario elevar la masa a mayores alturas o utilizar cargas más pesadas, lo que presenta desafíos adicionales en términos de infraestructura y control.\par
            
        \subsection{Eficiencia del Motor/Generador}
        
            Otro desafío importante fue la eficiencia del motor/generador. El prototipo utiliza un motor con una potencia de 195W y una reducción significativa de la velocidad, de 8100 RPM a 11 RPM. Aunque este sistema ha demostrado ser funcional para el prototipo, las pérdidas de energía durante la conversión de energía potencial en energía eléctrica siguen siendo un problema. Los motores utilizados para la generación de energía en aplicaciones industriales suelen tener una mayor eficiencia, lo que sugiere que, para futuras versiones del proyecto, sería necesario invertir en motores de mejor calidad y reducir las fricciones mecánicas en el sistema.\par
            
        \subsection{Infraestructura Limitada}
            El espacio disponible y los materiales utilizados también representaron una limitación. Nuestro prototipo fue diseñado y construido con materiales reciclados y reutilizados, lo cual tiene un impacto positivo en el costo y la sostenibilidad del proyecto. Sin embargo, para un sistema comercialmente viable, la infraestructura debe ser más robusta y estar diseñada específicamente para soportar grandes cargas durante largos períodos de tiempo sin riesgo de deterioro estructural.\par
    
    \section{Próximos Pasos}
    
        A pesar de las limitaciones mencionadas, el proyecto \textcolor{dark_violet}{GraviCap} tiene un enorme potencial para ser escalado y mejorado, no solo a nivel técnico, sino también en cuanto a su impacto en el almacenamiento de energía renovable. Basándonos en lo aprendido durante el desarrollo del prototipo, los próximos pasos a seguir incluirán tanto mejoras en el diseño del sistema como la implementación de nuevas tecnologías para aumentar su eficiencia y capacidad.\par
        
        \subsection{Escalado del Sistema}
        
            El paso más inmediato es aumentar la escala del sistema. Esto implicará la construcción de una versión más grande del prototipo, utilizando masas más pesadas y elevándolas a mayores alturas para incrementar la capacidad de almacenamiento de energía. Además, se considerarán nuevas configuraciones que permitan optimizar el uso del espacio y mejorar la conversión de energía potencial en electricidad.\par
            
        \subsection{Optimización del Motor/Generador}
        
            Se buscarán motores y generadores más eficientes para maximizar la cantidad de energía que se puede recuperar durante el proceso de descarga. Esto podría incluir la implementación de motores de última generación, con menores pérdidas por fricción y una mayor capacidad de conversión de energía mecánica en eléctrica. También se evaluarán nuevas configuraciones de poleas y sistemas de transmisión para reducir las pérdidas mecánicas en el sistema.\par
        
        \subsection{Ampliación de Aplicaciones}
        
            Un aspecto clave del futuro de \textcolor{dark_violet}{GraviCap} será su adaptabilidad a diferentes escenarios de uso. Aunque el prototipo actual ha sido desarrollado principalmente como una demostración a pequeña escala, el sistema podría adaptarse a necesidades específicas, como el almacenamiento de energía en zonas rurales sin acceso a la red eléctrica o como respaldo energético para sistemas solares. A medida que el proyecto crezca, se buscará que la tecnología sea accesible y económicamente viable para su implementación a distintas escalas.\par
        
        \subsection{Pruebas y Validación}
            Finalmente, es esencial que el sistema sea sometido a pruebas rigurosas en condiciones reales. Se planifican pruebas adicionales para evaluar el rendimiento del sistema en diferentes entornos, considerando factores como la temperatura, la humedad y las condiciones de carga variables. Estas pruebas ayudarán a identificar cualquier posible fallo y permitirán ajustar el diseño antes de implementar el sistema a mayor escala.\par