\chapter{Bibliografía utilizada}
    
        \section{Cargador MPPT}
        
            \begin{itemize} [label=•]
                \setlength{\itemindent}{3em}
                \item Guía en la programación (\href{https://electronoobs.com/eng_arduino_tut133.php}{https://electronoobs.com/eng\_arduino\_tut133.php})
            \end{itemize}

        \section{Software}
        
            \begin{itemize} [label=•]
                \setlength{\itemindent}{3em}
                \item Programación en PlatformIO (Página Web: (\href{https://docs.platformio.org/en/latest/}{https://docs.platformio.org/en/latest/})
                \item Documentación FreeRTOS (Página Web: \href{https://www.freertos.org}{https://www.freertos.org})
                \item Pico SDK (Repositorio: \href{https://github.com/raspberrypi/pico-sdk/tree/master}{https://github.com/raspberrypi/pico-sdk/tree/master})
            \end{itemize}
            
        \section{Página Web}
        
            \begin{itemize} [label=•]
                \setlength{\itemindent}{3em}
                \item FreeCodeCamp (Sitio Web: \href{https://www.freecodecamp.org/}{https://www.freecodecamp.org/})
                \item W3Schools Online Web Tutorials (Sitio Web: \href{https://www.w3schools.com/}{https://www.w3schools.com/})
                \item Stack Overflow (Sitio Web: \href{https://stackoverflow.com/}{https://stackoverflow.com/})
                \item How To Create Working Contact Form Using HTLM \& CSS Recieve Contact From Data on Email (Enlace: \href{https://youtu.be/-HeadgoqJ7A}{https://youtu.be/-HeadgoqJ7A})
            \end{itemize}
            
        \section{Aplicación}
        
            \begin{itemize} [label=•]
                \setlength{\itemindent}{3em}
                \item MarvelApp (Sitio Web: \href{https://marvelapp.com/}{https://marvelapp.com/})
                \item Ionic (Sitio Web: \href{https://ionicframework.com/}{https://ionicframework.com/})
                \item FreeCodeCamp (Sitio Web: \href{https://www.freecodecamp.org/}{https://www.freecodecamp.org/})
                \item Charts.js (Sitio Web: \href{https://www.chartjs.org/docs/latest/}{https://www.chartjs.org/docs/latest/})
                \item Android Studio (Sitio Web: \href{https://developer.android.com/?hl=es-419}{https://developer.android.com/?hl=es-419})
            \end{itemize}
            
        \section{Documentación}
            \begin{itemize} [label=•]
                \setlength{\itemindent}{3em}
                \item CTAN (Sitio Web: \href{https://ctan.org}{https://ctan.org})
                \item geometry (\href{https://ctan.org/pkg/geometry}{https://ctan.org/pkg/geometry})
                \item hyperref (\href{https://ctan.org/pkg/hyperref}{https://ctan.org/pkg/hyperref})
                \item babel (\href{https://ctan.org/pkg/babel}{https://ctan.org/pkg/babel})
                \item tocloft (\href{https://ctan.org/pkg/hyperref}{https://ctan.org/pkg/tocloft})
                \item carlito (\href{https://ctan.org/pkg/carlito}{https://ctan.org/pkg/carlito})
                \item inputenc (\href{https://ctan.org/pkg/inputenc}{https://ctan.org/pkg/inputenc})
                \item setspace (\href{https://ctan.org/pkg/setspace}{https://ctan.org/pkg/setspace})
                \item enumitem (\href{https://ctan.org/pkg/enumitem}{https://ctan.org/pkg/enumitem})
                \item ragged2e (\href{https://ctan.org/pkg/ragged2e}{https://ctan.org/pkg/ragged2e})
                \item titlesec (\href{https://ctan.org/pkg/titlesec}{https://ctan.org/pkg/titlesec})
                \item fancyhdr (\href{https://ctan.org/pkg/fancyhdr}{https://ctan.org/pkg/fancyhdr})
                \item xcolor (\href{https://ctan.org/pkg/xcolor}{https://ctan.org/pkg/xcolor})
                \item appendix (\href{https://ctan.org/pkg/appendix}{https://ctan.org/pkg/appendix})
                \item graphicx (\href{https://ctan.org/pkg/graphicx}{https://ctan.org/pkg/graphicx})
                \item rotating (\href{https://ctan.org/pkg/rotating}{https://ctan.org/pkg/rotating})
                \item adjustbox (\href{https://ctan.org/pkg/adjustbox}{https://ctan.org/pkg/adjustbox})
                \item pdflcape (\href{https://ctan.org/pkg/pdflscape}{https://ctan.org/pkg/pdflscape})
                \item subcaption (\href{https://ctan.org/pkg/subcaption}{https://ctan.org/pkg/subcaption})
                \item float (\href{https://ctan.org/pkg/float}{https://ctan.org/pkg/float})
                \item array (\href{https://ctan.org/pkg/array}{https://ctan.org/pkg/array})
                \item tabularx (\href{https://ctan.org/pkg/tabularx}{https://ctan.org/pkg/tabularx})
                \item tabularray (\href{https://ctan.org/pkg/tabularray}{https://ctan.org/pkg/tabularray})
                \item cancel (\href{https://ctan.org/pkg/cancel}{https://ctan.org/pkg/cancel})
                \item listings (\href{https://ctan.org/pkg/listings}{https://ctan.org/pkg/listings})
                \item Tex Stack Exchange (Sitio Web: \href{https://tex.stackexchange.com}{https://tex.stackexchange.com})
                \item Overleaf (Sitio Web: \href{https://www.overleaf.com/}{https://www.overleaf.com/})
            \end{itemize}
            
        \section{Investigación}
            \begin{itemize} [label=•]
                \setlength{\itemindent}{3em}
                \item Cynthia Santos Gómez. \guillemotleft Almacenamiento de Energía Potencial Gravitatoria mediante el desplazamiento de bloques sólidos por ferrocarril. Proyecto ARES.\guillemotright. En: Universidad de Sevilla (2017).
                \item Santiago Blas Marreros, Abelardo Cajaleón Alcántara, Micaela Cajaleón Alcántara, Peruska Pareja Madera, José Sánchez León Velarde, Abel Yucra Palacios, Alberto Huiman Cruz. \guillemotleft Situación de manejo de las baterías de plomo ácido en el Perú\guillemotright. En: Revista del Instituto de investigación de la Facultad de minas, metalurgia y ciencias geográficas de la Universidad Nacional Mayor de San Marcos (2022).
                \item Ing. Godelia Canchari Silverio, Ing. Oswaldo Ortiz Sanchez.\guillemotleft SISTEMA DE GESTIÓN DE RESIDUOS PELIGROSOS (PILAS Y BATERIAS) EN LA FACULTAD DE INGENIERIA GEOLOGICA, MINERA, METALURGICA Y GEOGRAFICA DE LA UNIVERSIDAD NACIONAL MAYOR DE SAN MARCOS \guillemotright. En: Revista del Instituto de investigación de la Facultad de minas, metalurgia y ciencias geográficas Volumen 13 (2010).
            \end{itemize}